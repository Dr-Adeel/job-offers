\documentclass{article}

% Demande de stage Master laboratoire
% version 1 du 26 ocotbre 2020

\usepackage{graphicx}
\usepackage{amsfonts}
\usepackage{amsmath}
\usepackage{algorithmic,algorithm} 
\usepackage{amssymb}

\setlength{\topmargin}{0pt}
\setlength{\headheight}{0pt}
\setlength{\headsep}{0pt}
\setlength{\textheight}{650pt}
\setlength{\footskip}{0pt}

\setlength{\oddsidemargin}{-20pt}
\setlength{\textwidth}{460pt}
\setlength{\marginparsep}{5pt}
\setlength{\marginparwidth}{0pt}

\setlength{\evensidemargin}{0pt}


%=======================================================

%=======================================================
\begin{document}

%=======================================================
%Titre 

\begin{center}
\includegraphics[width=4cm]{logo_lisic.jpg}\hfill \includegraphics[width=2.5cm]{logo-ulco-16.png}\\

\vspace{1.5cm}
\textbf{\Large{Offre Stage Recherche - 2026}}
\vspace{2cm}
\end{center}

%=======================================================
\section{Title : Towards an Explainable Conversational Recommendation System using LLMs: Integrating Embeddings, XAI, and Chatbots}

%=======================================================
\section{Supervisors : Adeel Ahmad, Amani Braham, Mourad Bouneffa}
Contact : adeel.ahmad@univ-littoral.fr


%=======================================================
\section{Duration : 4-6 mois}

\section{Description of the subject :}
Building upon the work carried out this year \cite{choaib2025iot, ahmad2025rapport}, which focused on developing an intelligent conversational system for assisting with CPS configuration, this internship proposes an extension aimed at enhancing the intelligence of the recommendation engine and its explainability.

Chatbots and Large Language Models (LLMs) have generated growing interest due to their ability to understand user intent and formulate interactive and personalized responses. The integration of these technologies enables the envisioning of a new generation of conversational recommendation systems, capable of adapting in real-time to the context and evolving needs expressed during the interaction. From this perspective, this project aims to develop an LLM-based recommendation system, capable of understanding natural language queries and producing relevant recommendations through fine-tuning and vector representations (embeddings). The work will first involve defining and preparing an application domain by building a structured dataset comprising components, scenarios, and performance criteria, then generating the necessary embeddings to populate a vector database. The intern will then conduct a comparative analysis of available open-source language models, select the one offering the best compromise between performance and the laboratory's technical constraints, and implement realistic adaptation methods such as LoRA, instruction tuning, or a RAG approach. Subsequently, they will develop an explainable recommendation engine capable of establishing a semantic match between user queries and the vector database, selecting items based on explicit criteria such as compatibility, accuracy, or cost; and generating explanations based on vector similarity, business rules derived from the team's current work, and technical metadata. Although models like BERT and its derivatives offer excellent performance in natural language understanding, their use in recommendation systems often remains opaque. The originality of this project thus lies in the integration of advanced Explainable AI (XAI) mechanisms to make the system's operation interpretable without sacrificing its performance. The final objective is to propose a system capable of generating recommendations that are not only relevant but also accompanied by understandable explanations, thereby enhancing transparency, trust, and user adoption.


%=======================================================
\section{Context and Objectives of the Proposal:}
Initial work \cite{choaib2025iot, choaib2023automated} enabled the design of a chatbot capable of collecting user needs and the integration of a recommendation engine based on a CPS knowledge base. The project uses embeddings and a vector search engine to identify the best configurations. This internship aims to integrate a specialized LLM to obtain a system capable of providing accurate recommendations while ensuring transparency regarding the reasons behind the proposed suggestions. The main objective of the internship is to develop an explainable recommendation system based on an LLM model and integrated into the existing chatbot. The system must leverage a vector database, built from embeddings, to identify the most relevant items and generate optimized recommendations consistent with the target domains. It should rely on the semantics of the embeddings and on explicit selection criteria to improve user understanding and trust. This system will be integrated into the pipeline already in place within the team, including the chatbot, the recommendation engine, and the CPS knowledge bases, in direct continuity with the scientific priorities of the IC/SysReIC team at LISIC.

\section{References: }

\begin{thebibliography}{9}
\bibitem{choaib2025iot} 
Mohammad Choaib, Moncef Garouani, Mourad Mohamed Bouneffa, Adeel Ahmad. 
\textit{IoT-AID: Leveraging XAI for Conversational Recommendations in Cyber-Physical Systems}. 
27th International Conference on Enterprise Information Systems, INSTICC, Apr 2025, Porto, Portugal. 
pp.671-679, ⟨10.5220/0013497100003929⟩. ⟨hal-05068872⟩

\bibitem{ahmad2025rapport}
Adeel Ahmad, NAJIB ALAOUI Chaima.
\textit{Développement d'un système de recommandation explicable pour les systèmes cyber-physiques (CPS), intégrant un chatbot intelligent, la sémantique des données et des bases de données vectorielles, afin de proposer des configurations personnalisées et transparentes adaptées aux besoins des utilisateurs.}
Rapport de Stage Assistant Ingénieur, du 15/05/2025 au 12/09/2025 LISIC, ULCO

\bibitem{choaib2023automated}
Mohammad Choaib, Moncef Garouani, Mourad Mohamed Bouneffa, Nicolas Waldhoff, Adeel Ahmad, et al.
\textit{Automated Decision Support Framework for IoT: Towards a Cyber Physical Recommendation System.}
25th International Conference on Enterprise Information Systems (ICEIS 2023), Apr 2023, Prague, Czech Republic.
pp.365-373, ⟨10.5220/0011848900003467⟩. ⟨hal-04129196⟩
\end{thebibliography}

%1. Mohammad Choaib, Moncef Garouani, Mourad Mohamed Bouneffa, Adeel Ahmad. IoT-AID: Leveraging XAI for Conversational Recommendations in Cyber-Physical Systems. 27th International Conference on Enterprise Information Systems, INSTICC : Institute for Systems and Technologies of Information, Control and Communication, Apr 2025, Porto, Portugal. pp.671-679, ⟨10.5220/0013497100003929⟩. ⟨hal-05068872⟩
%
%2. Adeel Ahmad, NAJIB ALAOUI Chaima, ‘Développement d’un système de recommandation explicable pour les systèmes cyber-physiques (CPS), intégrant un chatbot intelligent, la sémantique des données et des bases de données vectorielles, afin de proposer des configurations personnalisées et transparentes adaptées aux besoins des utilisateurs.’, Rapport de Stage Assistant Ingénieur, du 15/05/2025 au 12/09/2025 LISIC, ULCO
%
%3. Mohammad Choaib, Moncef Garouani, Mourad Mohamed Bouneffa, Nicolas Waldhoff, Adeel Ahmad, et al.. Automated Decision Support Framework for IoT: Towards a Cyber Physical Recommendation System. 25th International Conference on Enterprise Information Systems (ICEIS 2023), Apr 2023, Prague, Czech Republic. pp.365-373, ⟨10.5220/0011848900003467⟩. ⟨hal-04129196⟩

\end{document}

 
