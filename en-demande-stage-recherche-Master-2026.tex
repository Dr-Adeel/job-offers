\documentclass{article}

% Demande de stage Master laboratoire
% version 1 du 26 ocotbre 2020

\usepackage{graphicx}
\usepackage{amsfonts}
\usepackage{amsmath}
\usepackage{algorithmic,algorithm} 
\usepackage{amssymb}

\setlength{\topmargin}{0pt}
\setlength{\headheight}{0pt}
\setlength{\headsep}{0pt}
\setlength{\textheight}{650pt}
\setlength{\footskip}{0pt}

\setlength{\oddsidemargin}{-20pt}
\setlength{\textwidth}{460pt}
\setlength{\marginparsep}{5pt}
\setlength{\marginparwidth}{0pt}

\setlength{\evensidemargin}{0pt}


%=======================================================

%=======================================================
\begin{document}

%=======================================================
%Titre 

\begin{center}
\includegraphics[width=4cm]{logo_lisic.jpg}\hfill \includegraphics[width=2.5cm]{logo-ulco-16.png}\\

\vspace{1.5cm}
\textbf{\Large{Demande Stage Master Recherche 2026}}
\vspace{2cm}
\end{center}

%=======================================================
\section{Titre : Vers un Système de Recommandation Conversationnel Explicable Basé sur un LLM : Intégration d’Embeddings, XAI et Chatbot}

%=======================================================
\section{Encadrant(e)(s) : Adeel Ahmad, Amani Braham, Mourad Bouneffa}


%=======================================================
\section{Dur\'ee : 4-6 mois}

\section{Description du sujet :}
Dans la continuité des travaux menés cette année \cite{choaib2025iot,ahmad2025rapport}, portant sur le développement d’un système conversationnel intelligent pour l’assistance à la configuration de systèmes CPS, ce stage propose une extension visant à renforcer l’intelligence du moteur de recommandation et son explicabilité.

Les chatbots et les modèles de langage de grande taille (LLM) ont suscité un intérêt croissant grâce à leur capacité à comprendre les intentions des utilisateurs et à formuler des réponses interactives et personnalisées. L’intégration de ces technologies permet d’envisager une nouvelle génération de systèmes de recommandation conversationnels, capables de s’adapter en temps réel au contexte et aux besoins évolutifs exprimés au fil de l’interaction. Dans cette perspective, ce projet vise à développer un système de recommandation basé sur un LLM, capable de comprendre les requêtes formulées en langage naturel et de produire des recommandations pertinentes grâce au recours au fine-tuning et aux représentations vectorielles (embeddings). Le travail consistera d’abord à définir et préparer un domaine applicatif en construisant un dataset structuré comprenant des composants, des scénarios et des critères de performance, puis en générant les embeddings nécessaires à l’alimentation d’une base vectorielle. Le stagiaire mènera ensuite une analyse comparative des modèles de langage open-source disponibles, sélectionnera celui présentant le meilleur compromis entre performance et contraintes techniques du laboratoire, et mettra en œuvre des méthodes réalistes d’adaptation telles que LoRA, l’instruction tuning ou une approche RAG. Par la suite, il développera un moteur de recommandation explicable capable d’établir une correspondance sémantique entre les requêtes utilisateur et la base vectorielle, de sélectionner les items selon des critères explicites, tels que la compatibilité, la précision ou le coût ; et de générer des explications fondées sur la similarité vectorielle, les règles métiers issues des travaux actuels de l’équipe et les métadonnées techniques. Bien que des modèles comme BERT et ses dérivés offrent d’excellentes performances en compréhension du langage naturel, leur utilisation dans les systèmes de recommandation demeure souvent opaque. L’originalité de ce projet réside ainsi dans l’intégration de mécanismes avancés d’IA explicable afin de rendre le fonctionnement du système interprétable, sans sacrifier sa performance. L’objectif final est de proposer un système capable de générer des recommandations non seulement pertinentes, mais également accompagnées d’explications compréhensibles, renforçant ainsi la transparence, la confiance et l’adoption par les utilisateurs.



%=======================================================
\section{Contexte et objectifs de la demande}
Les premiers travaux ont permis \cite{choaib2025iot,choaib2023automated}, la conception d’un chatbot capable de collecter les besoins utilisateurs et l’intégration d’un moteur de recommandation basé sur une base de connaissances CPS. Le projet utilise d’embeddings et d’un moteur de recherche vectorielle pour identifier les meilleures configurations. Ce stage vise à intégrer un LLM spécialisé, afin d’obtenir un système capable de fournir des recommandations précises, tout en assurant une transparence sur les raisons ayant conduit aux suggestions proposées. L’objectif principal du stage est de développer un système de recommandation explicable basé sur un modèle LLM et intégré dans le chatbot existant. Le système devra exploiter une base de données vectorielle, construite à partir d’embeddings, afin d’identifier les éléments les plus pertinents et de générer des recommandations optimisées, en cohérence avec les domaines ciblés. Il devra s’appuyer sur la sémantique des embeddings et sur des critères explicites de sélection, afin d’améliorer la compréhension et la confiance de l’utilisateur. Ce système sera intégré dans le pipeline déjà en place au sein de l’équipe, incluant le chatbot, le moteur de recommandation et les bases de connaissances CPS, en continuité directe avec les priorités scientifiques de l’équipe IC/SysReIC du LISIC.

\section{Références: }

\begin{thebibliography}{9}
\bibitem{choaib2025iot} 
Mohammad Choaib, Moncef Garouani, Mourad Mohamed Bouneffa, Adeel Ahmad. 
\textit{IoT-AID: Leveraging XAI for Conversational Recommendations in Cyber-Physical Systems}. 
27th International Conference on Enterprise Information Systems, INSTICC, Apr 2025, Porto, Portugal. 
pp.671-679, ⟨10.5220/0013497100003929⟩. ⟨hal-05068872⟩

\bibitem{ahmad2025rapport}
Adeel Ahmad, NAJIB ALAOUI Chaima.
\textit{Développement d'un système de recommandation explicable pour les systèmes cyber-physiques (CPS), intégrant un chatbot intelligent, la sémantique des données et des bases de données vectorielles, afin de proposer des configurations personnalisées et transparentes adaptées aux besoins des utilisateurs.}
Rapport de Stage Assistant Ingénieur, du 15/05/2025 au 12/09/2025 LISIC, ULCO

\bibitem{choaib2023automated}
Mohammad Choaib, Moncef Garouani, Mourad Mohamed Bouneffa, Nicolas Waldhoff, Adeel Ahmad, et al.
\textit{Automated Decision Support Framework for IoT: Towards a Cyber Physical Recommendation System.}
25th International Conference on Enterprise Information Systems (ICEIS 2023), Apr 2023, Prague, Czech Republic.
pp.365-373, ⟨10.5220/0011848900003467⟩. ⟨hal-04129196⟩
\end{thebibliography}

%1. Mohammad Choaib, Moncef Garouani, Mourad Mohamed Bouneffa, Adeel Ahmad. IoT-AID: Leveraging XAI for Conversational Recommendations in Cyber-Physical Systems. 27th International Conference on Enterprise Information Systems, INSTICC : Institute for Systems and Technologies of Information, Control and Communication, Apr 2025, Porto, Portugal. pp.671-679, ⟨10.5220/0013497100003929⟩. ⟨hal-05068872⟩
%
%2. Adeel Ahmad, NAJIB ALAOUI Chaima, ‘Développement d’un système de recommandation explicable pour les systèmes cyber-physiques (CPS), intégrant un chatbot intelligent, la sémantique des données et des bases de données vectorielles, afin de proposer des configurations personnalisées et transparentes adaptées aux besoins des utilisateurs.’, Rapport de Stage Assistant Ingénieur, du 15/05/2025 au 12/09/2025 LISIC, ULCO
%
%3. Mohammad Choaib, Moncef Garouani, Mourad Mohamed Bouneffa, Nicolas Waldhoff, Adeel Ahmad, et al.. Automated Decision Support Framework for IoT: Towards a Cyber Physical Recommendation System. 25th International Conference on Enterprise Information Systems (ICEIS 2023), Apr 2023, Prague, Czech Republic. pp.365-373, ⟨10.5220/0011848900003467⟩. ⟨hal-04129196⟩

\end{document}

 
